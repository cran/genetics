% $Id: genetics_article.tex,v 1.6 2003/02/03 16:13:56 warnesgr Exp $
%
% $Log: genetics_article.tex,v $
% Revision 1.6  2003/02/03 16:13:56  warnesgr
% - Fixed typos and R CMD check warnings.
% - Updated version number
% - Removed 'data' directory to fix new R CMD check warning.
%
% Revision 1.5  2002/11/27 15:32:20  warnesgr
% Correct spelling errors and typos.
%
% Revision 1.4  2002/06/27 18:46:56  warnesgr
%
% - More revisions.  Hopefully last set before submission to publication
%   review.
%
% Revision 1.3  2002/06/25 21:38:57  warnesgr
%
% - Fixed syntax errors
% - Some reorganization
%
% Revision 1.2  2002/06/19 10:34:54  warnesgr
%
% Much enhancement, including addition of example section.
%
%

\documentclass{report}
\usepackage{Rnews}
\begin{document}

\author{by Gregory R. Warnes}
\title{The genetics package}
\subtitle{Utilities for handling genetic data}

\maketitle

\section{Purpose}

The genetics package provides classes and methods for handling genetic
data.  Friedrich ('Fritz') Leisch and I collaborated on the design of
the package.  I was motivated by the desire to provide a natural way
to include single-locus genetic variables in statistical models.
Fritz also wanted to support multiple genetic changes spread across
one or more genes.  While my goal has largely been realized, more work
is necessary to fully support Fritz's goal.

\section{Current Status}

As of version 0.6.5 the library includes classes and methods for
creating, representing, and manipulating genotypes (unordered allele
pairs) and haplotypes (ordered allele pairs).  Genotypes and
haplotypes can be annotated with chromosome, locus, gene, and marker
information. Utility functions compute genotype and allele
frequencies, flag homozygotes or heterozygotes, flag allele carriers
of certain alleles, count the number of a specific allele carried by
an individual, extract one or both alleles, and test Hardy-Weinberg
equilibrium.  These functions make it easy to create and use
single-locus genetic information. In addition, a function is provide
to compute the probability of failing to observe all alleles with a
given true frequency.

Creating variables representing genotype in a statistical package
often requires considerable string manipulation.  The code for the
\code{genotype} function has been designed to remove this requirement.
It allows alleles pairs to be specified in four ways:

\begin{itemize} 
\item A single vector with a character separator:
  {\small
\begin{verbatim} 
  g1 <- genotype( c('A/A','A/C','C/C','C/A',
                     NA,'A/A','A/C','A/C') )
  g3 <- genotype( c('A A','A C','C C','C A',
                    '','A A','A C','A C'), 
                  sep=' ', remove.spaces=F)
\end{verbatim}
}  


\item A single vector with a positional separator
  {\small
\begin{verbatim}
  g2 <- genotype( c('AA','AC','CC','CA','',
                    'AA','AC','AC'), sep=1 )
\end{verbatim}
}  


\item Two separate vectors
  {\small
\begin{verbatim}
  g4 <- genotype( 
          c('A','A','C','C','','A','A','A'),
          c('A','C','C','A','','A','C','C')
          )
\end{verbatim}
}  

\item A dataframe or matrix with two columns
  {\small
\begin{verbatim}
  gm <- cbind( 
          c('A','A','C','C','','A','A','A'),
          c('A','C','C','A','','A','C','C') ) 
  g5 <- genotype( gm )
\end{verbatim}
}  
\end{itemize}

A second difficulty with variables representing genotypes is the need
to extract different information at different times.  Each of the
three basic ways of modeling the effect the allele combinations is
supported by the \code{genetics} package:
\begin{description}
\item[categorical] Each allele combination acts differently.
  
  This situation is handled by entering the \code{genotype} variable without
  modification into a model.  In this case, it will be treated as a
  factor:

{\small
\begin{verbatim}
lm( outcome ~ genotype.var + confounder )
\end{verbatim}
}
  
\item[additive] The effect depends on the number of copies of a
  specific allele (0, 1, or 2).
  
  The function \code{allele.count( gene, allele )} returns the number
  of copies of the specified allele.
  
{\small
\begin{verbatim}
lm( outcome ~ allele.count(genotype.var,'A') 
              + confounder )
\end{verbatim}
}
  
\item[dominant/recessive] The effect depends only on the presence or
  absence of a specific allele.
  
  The function \code{carrier( gene, allele )} returns a boolean flag
  if the specified allele is present:

{\small
\begin{verbatim}
lm( outcome ~ carrier(genotype.var,'A') 
              + confounder )
\end{verbatim}
}

\end{description}




\section{Implementation}

The basic functionality of the \code{genetics} package is provided by
the \code{genotype} class and the \code{haplotype} class, which is a
simple extension of the former.  In designing the \code{genotype}
class, we had several goals.  First, we wanted to be able to
manipulate both alleles as a single variable.  Second, we needed a
clean way of accessing the individual alleles.  Third, a genotype
variable should be able to be stored in dataframes as they are
currently implemented in R.  Fourth, storage should be
space-efficient.

After considering several potential implementations, we chose to
implement the genotype class as an extension to the in-built factor
variable type with additional information stored in attributes.
Genotype objects are stored as factors and have the class list
\code{c("genotype","factor")}.  The names of the factor levels are
constructed as \code{paste(allele1,"/",allele2,sep="")}.  Since most
genotyping methods do not indicate which allele comes from which
member of a chromosome pair, the alleles for each individual are
placed in a consistent order controlled by the \code{reorder}
argument.  In cases when the allele order is informative, the
\code{haplotype} class, which preserves the allele order, should be
used instead.

The set of allele names is stored in the attribute
\code{allele.names}.  A translation table from the factor levels to
the names of each of the two alleles is stored in the attribute
\code{allele.map}.  This map is a two column character matrix with one
row per factor level.  The columns provide the individual alleles for
each factor level.  Accesing the individual alleles, as performed by the \code{allele} function, is accomplished by simply indexing into this table,
\begin{verbatim}
allele.x <- attrib(x,"allele.map") 
alleles.x[genotype.var,which]
\end{verbatim}
where \code{which} is \code{1}, \code{2}, or \code{c(1,2)} as
appropriate.

Finally, there is often additional meta-information associated with a
genotype.  The functions \code{locus}, \code{gene}, and \code{marker}
create objects to store information, respectively, about genetic loci,
genes, and markers.  Any of these objects can be included as part of a
genotype object using the \code{locus} argument, which creates a
\code{locus} attribute in the genotype object.  The print and summary
functions for genotype objects properly display this information when
it is present.

This implementation of the genotype class met our four design goals
and offered an additional benefit: the default behavior for factors is
similar to the desired behavior for genotypes.  Consequently,
relatively few additional methods needed to written.  Further, in the
absence of the genetics package, the information stored in genotype
objects is still accessible in a reasonable way.

The \code{genotype} class is accompanied by a full complement of
helper methods for standard R operators ( \code{[]}, \code{[<-},
\code{==}, etc. ) and object methods ( \code{summary}, \code{print},
\code{is.genotype}, \code{as.genotype}, etc. ).  The \code{genetics}
package provides the additional functions:

\begin{description}
  
\item[HWE.test] Estimates disequilibrium parameter and test the null
  hypothesis that Hardy-Weinberg equilibrium holds.
  
\item[allele] Extracts individual alleles.
  matrix.

\item[allele.names] Extracts the set of allele names.
  
\item[homozygote] Creates a logical vector inidicating whether both
  alleles of each observation are the same.
  
\item[heterozygote] Creates a logical vector indicating whether the
  alleles of each observation differ.
  
\item[carrier] Creates a logical vector indicating whether
  the specified alleles are present.
  
\item[allele.count] Returns the number of copies of the specified
  alleles carried by each observation.
  
\item[getlocus] Extracts locus, gene, or marker information.

\end{description}

For complete details on the objects and functions provided by the
\code{genetics} package, please see the help pages \code{?genotype},
\code{?HWE.test}, \code{?homozygote}, \code{?locus}, and
\code{?HWE.test} or the corresponding auto-generated documentation.

\section{Example}

Here is a partial session using tools from the genotype package to
examine the features of 3 simulated markers and thier relationships
with a continuous outcome:

{\small
\begin{verbatim}
> library(genetics)

Attaching package `genetics':


        The following object(s) are masked from package:base :

         as.factor 

> ## Create a sample dataset with 3 SNP markers
> 
> g1 <- sample( x=c('C/C', 'C/T', 'T/T'), 
+               prob=c(.6,.2,.2), 20, replace=T)
> g2 <- sample( x=c('A/A', 'A/G', 'G/G'), 
+               prob=c(.6,.1,.5), 20, replace=T)
> g3 <- sample( x=c('C/C', 'C/T', 'T/T'), 
+               prob=c(.2,.4, 4), 20, replace=T)
> 
> y <- rnorm(20) + (g1=='C/C') + 
+      0.25 * (g2=='A/A' | g2=='A/G')
> 
> ## Form into a data frame
> data <- data.frame( y, g1, g2, g3)
> 
> # Create marker labels for the data 

[...]

> a1691g  <- marker(name="A1691G",
+                  type="SNP",
+                  locus.name="MBP2",
+                  chromosome=9, 
+                  arm="q", 
+                  index.start=35,
+                  bp.start=1691,
+                  relative.to="intron 1")
> 
> 

[...]

> 
> data$g1 <- genotype(data$g1, locus=c104t)
> data$g2 <- genotype(data$g2, locus=a1691g)
> data$g3 <- genotype(data$g3, locus=c2249t)
> 
> data
              y  g1  g2  g3
1  -0.084796634 T/T G/G T/C
2   1.454537575 C/C G/G T/T
3  -0.899625344 T/T G/G T/T
4  -1.980679630 C/T A/A T/T
5   0.231087028 C/T A/A T/T
6   2.588083646 C/C A/A T/C
7   0.209338731 C/C A/A T/T
8   1.435823157 C/T G/G T/T
9  -0.078796949 C/C G/G T/T
10 -2.091110058 C/T A/A T/T
11 -0.842655686 C/T G/G T/T
12  1.316828279 C/C G/G T/T
13  0.470126626 C/T A/A T/T
14 -0.364828611 T/T G/A T/T
15 -0.002438264 C/T A/A T/C
16  0.949432430 C/C G/G T/T
17 -0.096626850 C/T G/A T/T
18  1.065637984 T/T A/A T/T
19  0.817213289 C/C A/A T/T
20  0.644714638 C/T G/G T/T
> 
> summary(data$g2)

Marker: MBP2:A1691G (9q35:1691) Type: SNP

Allele Frequency:
  Count Proportion
A    20        0.5
G    20        0.5

Genotype Frequency:
    Count Proportion
A/A     9       0.45
G/A     2       0.10
G/G     9       0.45

> HWE.test(data$g2)

Test for Hardy-Weinberg-Equilibrium

Call: 
HWE.test.genotype(x = data$g2)

Disequlibrium Estimate (D-hat):
        Observed Expected Obs-Exp D-hat
G/G            9        5       4   0.2
G/A            2       10      -8  -0.2
A/A            9        5       4   0.2
Overall       20       NA      NA   0.2

Significance Test:

        Pearson's Chi-squared test with simulated
        p-value (based on 10000 replicates)

data:  data$g2 
X-squared = 12.8, df = NA, p-value = 0.0011



>
> summary(lm( y ~ homozygote(g1,'C') +
                allele.count(g2, 'G') +
+                 g3, data=data))
+ 
Call:
lm(formula = y ~ homozygote(g1, "C") + allele.count(g2, "G") + 
    g3, data = data)

Residuals:
    Min      1Q  Median      3Q     Max 
-1.6686 -0.6625 -0.0172  0.6973  1.6196 

Coefficients:
                        Estimate Std. Error t value Pr(>|t|)  
(Intercept)               0.3499     0.6229   0.562   0.5821  
homozygote(g1, "C")TRUE   1.2124     0.4778   2.537   0.0220 *
allele.count(g2, "G")     0.1193     0.2429   0.491   0.6298  
g3T/T                    -0.7724     0.6414  -1.204   0.2460  
---
Signif. codes:  0 `***' 0.001 `**' 0.01 `*' 0.05 `.' 0.1 ` ' 1 

Residual standard error: 1.013 on 16 degrees of freedom
Multiple R-Squared: 0.3405,     Adjusted R-squared: 0.2169 
F-statistic: 2.754 on 3 and 16 DF,  p-value: 0.07661 

\end{verbatim}
}


\section{Conclusion}

The \code{genetics} package has already proven useful in my work here
at Pfizer.  I hope that it will also be of user to others who need to
analyze genetic data.

In the future I expect to add functions to compute pairwise
disequilibrium, perform haplotype imputation, and generate standard
genetics plots.  I welcome comments and contributions.

\address{Gregory R. Warnes \\
        Pfizer Global Research and Development \\
        \emph{gregory\_r\_warnes$@$groton.pfizer.com} }  %%!!!%%


\end{multicols}  %%!!!%%

\end{document}
